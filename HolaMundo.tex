\documentclass{article}
\usepackage{graphicx} % Required for inserting images

\title{HolaMundo}
\author{Iván Pérez Angeles}
\date{Abril 2023}

\begin{document}

\maketitle

\section{Introducción}
\subsection{Primera subsección}
\subsubsection{Primera subsubsección}

\textbf{Estas son negrillas.} \textit{Esto es Italika}
Se escribe con cursivas y entre comillas las citas, por ejemplo el artículo 1 constitucional:
\subsubsection{Segunda subsubsección}
"\textit{ En los Estados Unidos Mexicanos todas las personas gozarán de los derechos humanos
reconocidos en esta Constitución y en los tratados internacionales de los que el Estado Mexicano sea
parte, así como de las garantías para su protección, cuyo ejercicio no podrá restringirse ni suspenderse,
salvo en los casos y bajo las condiciones que esta Constitución establece.}"
\subsection{Segunda subsección}
La Nación tiene una composición pluricultural sustentada originalmente en sus pueblos indígenas que
son aquellos que descienden de poblaciones que habitaban en el territorio actual del país al iniciarse la
colonización y que conservan sus propias instituciones sociales, económicas, culturales y políticas, o
parte de ellas.


\section{Estado del Arte}

\end{document}

